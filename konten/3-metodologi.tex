\chapter{METODOLOGI}

% Ubah konten-konten berikut sesuai dengan isi dari metodologi

\section{Metode yang Digunakan}

\lipsum[11]

% Contoh input gambar dengan format *.jpg
\begin{figure} [H] \centering
  % Nama dari file gambar yang diinputkan
  \includegraphics[scale=0.45]{gambar/blueprint.jpg}
  % Keterangan gambar yang diinputkan
  \caption{\emph{Blueprint} roket yang akan diuji coba \parencite{SpaceXBlueprint}}
  % Label referensi dari gambar yang diinputkan
  \label{fig:Blueprint}
\end{figure}

% Contoh penggunaan referensi dari gambar yang diinputkan
Pada \emph{blueprint} yang tertera di Gambar \ref{fig:Blueprint}. \lipsum[12]

\section{Bahan dan Alat yang Digunakan}

Beberapa bahan dan peralatan yang digunakan untuk mendukung penelitian ini antara lain:

\subsection{Perangkat Keras}
\lipsum[13]

\subsection{Perangkat Lunak}
\lipsum[14]

\section{Urutan Pelaksanaan Penelitian}

\lipsum[17]

\begin{lstlisting}[caption=Contoh algoritma paralel]
  ALGORITHM DistributedDataProcessing
  INPUT: dataset, cluster_nodes, processing_config
  OUTPUT: processed_results
  
  BEGIN
      // Initialize cluster configuration
      cluster = InitializeCluster(cluster_nodes)
      data_partitions = PartitionData(dataset, cluster.size)
      result_aggregator = CreateResultAggregator()
      
      // Distributed processing phase
      FOR EACH partition IN data_partitions DO PARALLEL
          node = cluster.GetAvailableNode()
          
          // Load balancing and fault tolerance
          WHILE NOT IsNodeHealthy(node) DO
              node = cluster.GetBackupNode(node)
              IF node == NULL THEN
                  THROW NodeFailureException
              END IF
          END WHILE
          
          // Process data on assigned node
          local_result = ProcessPartition(partition, node)
          
          // Apply filtering and transformation
          filtered_data = ApplyFilters(local_result, processing_config.filters)
          transformed_data = ApplyTransformations(filtered_data, processing_config.transforms)
          
          // Quality assurance and validation
          validation_result = ValidateResults(transformed_data)
          IF validation_result.is_valid THEN
              result_aggregator.AddResult(transformed_data, node.id)
          ELSE
              // Retry processing with different configuration
              retry_config = AdjustProcessingConfig(processing_config, validation_result.errors)
              retry_result = ProcessPartition(partition, node, retry_config)
              result_aggregator.AddResult(retry_result, node.id)
          END IF
      END FOR
      
      // Wait for all parallel processes to complete
      WaitForCompletion(cluster)
      
      // Aggregate and optimize final results
      aggregated_results = result_aggregator.AggregateResults()
      optimized_results = OptimizeResults(aggregated_results)
      
      // Cleanup cluster resources
      cluster.Cleanup()
      
      RETURN optimized_results
  END
\end{lstlisting}