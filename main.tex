% Pengaturan ukuran teks dan bentuk halaman satu sisi
\documentclass[12pt,oneside]{book}

% Pengaturan ukuran halaman dan margin
\usepackage[a4paper,top=30mm,left=30mm,right=20mm,bottom=25mm]{geometry}

% Pengaturan ukuran spasi
\usepackage[singlespacing]{setspace}

% Pengaturan caption untuk tabel
\usepackage{caption}

% Judul dokumen
\title{Proposal Tugas Akhir ITS}
\author{Musk, Elon Reeve}

% Pengaturan detail pada file PDF
\usepackage[pdfauthor={\@author},bookmarksnumbered,pdfborder={0 0 0},linktoc=all]{hyperref}


% Pengaturan ukuran indentasi
\setlength{\parindent}{2em}

% Package lainnya
\usepackage{changepage}
\usepackage{etoolbox} % Mengubah fungsi default

% Pengaturan jenis karakter
\usepackage[utf8]{inputenc}

\usepackage{url}
\usepackage[style=apa, backend=biber]{biblatex}
\usepackage{xurl}

\usepackage{enumitem} % Pembuatan list
\usepackage{lipsum} % Pembuatan template kalimat
\usepackage{graphicx} % Input gambar
\usepackage{longtable} % Pembuatan tabel
\usepackage[table,xcdraw]{xcolor} % Pewarnaan tabel
\usepackage{eso-pic} % Untuk menggunakan background image di halaman
\usepackage{txfonts} % Font times
\usepackage{changepage} % Pembuatan teks kolom
\usepackage{multicol} % Pembuatan kolom ganda
\usepackage{multirow} % Pembuatan baris ganda
\usepackage{tabularx} % Untuk mengatur kolom, seperti grid pada CSS
\usepackage{wrapfig}
\usepackage{float}
\usepackage[linesnumbered,ruled,vlined]{algorithm2e}

% Package untuk code listings
\usepackage{listings}
\usepackage{xcolor}

% Pengaturan format daftar isi, daftar gambar, dan daftar tabel
\usepackage[titles]{tocloft}
% Definisi command untuk list of listings
\newcommand{\cftloctitlefont}{}
\newcommand{\cftafterloctitle}{}
\setlength{\cftsecindent}{2em}
\setlength{\cftsubsecindent}{3em}
\setlength{\cftsubsubsecindent}{4em}
\setlength{\cftbeforechapskip}{1.5ex}
\setlength{\cftbeforesecskip}{1ex}
\setlength{\cftbeforesubsecskip}{1ex}
\setlength{\cftbeforesubsubsecskip}{1ex}
\setlength{\cftbeforetoctitleskip}{0cm}
\setlength{\cftbeforeloftitleskip}{4ex}
\setlength{\cftafterloftitleskip}{0cm}
\setlength{\cftbeforelottitleskip}{0cm}
\setlength{\cftfigindent}{0pt}
\setlength{\cfttabindent}{0pt}
\renewcommand{\cfttoctitlefont}{\hfill\Large\bfseries} % command untuk membuat heading bold dan besar
\renewcommand{\cftaftertoctitle}{\hfill}
\renewcommand{\cftloftitlefont}{\hfill\Large\bfseries}
\renewcommand{\cftafterloftitle}{\hfill}
\renewcommand{\cftlottitlefont}{\hfill\Large\bfseries}
\renewcommand{\cftafterlottitle}{\hfill}
\renewcommand{\cftloctitlefont}{\hfill\Large\bfseries}
\renewcommand{\cftafterloctitle}{\hfill}

% Definisi untuk "Hati ini sengaja dikosongkan"
\patchcmd{\cleardoublepage}{\hbox{}}{
  \thispagestyle{empty}
  \vspace*{\fill}
  \begin{center}\textit{[Halaman ini sengaja dikosongkan]}\end{center}
  \vfill}{}{}

% Pengaturan penomoran halaman
\usepackage{fancyhdr}
\fancyhf{}
\renewcommand{\headrulewidth}{0pt}
\pagestyle{fancy}
\fancyfoot[C,CO]{\thepage}
\patchcmd{\chapter}{plain}{fancy}{}{}
\patchcmd{\chapter}{empty}{plain}{}{}

% Pengaturan format judul bab
\usepackage{titlesec}
\renewcommand{\thesection}{\thechapter.\arabic{section}}
\titleformat{\chapter}[hang]{\centering\bfseries\large}{BAB\ \arabic{chapter}\ }{0ex}{\vspace{0ex}\centering}
\titleformat*{\section}{\large\bfseries}
\titleformat*{\subsection}{\normalsize\bfseries}
\titleformat*{\subsubsection}{\normalsize\bfseries}
\titlespacing{\chapter}{0ex}{0ex}{4ex}
\titlespacing{\section}{0ex}{1ex}{0ex}
\titlespacing{\subsection}{0ex}{0.5ex}{0ex}
\titlespacing{\subsubsection}{0ex}{0.5ex}{0ex}
\setcounter{secnumdepth}{4} % Untuk memberi penomoran pada \subsubsection
\setcounter{tocdepth}{4}

\counterwithin{figure}{chapter}
\counterwithin{table}{chapter}

% Mengganti figure dan table menjadi gambar dan tabel
\renewcommand{\figurename}{Gambar}
\renewcommand{\tablename}{Tabel}
\renewcommand{\lstlistingname}{Kode Sumber}

% Pengaturan styling untuk code listings
\lstset{
    basicstyle=\ttfamily\footnotesize,
    breaklines=true,
    breakatwhitespace=true,
    showstringspaces=false,
    frame=single,
    numbers=left,
    numberstyle=\tiny,
    stepnumber=1,
    numbersep=5pt,
    backgroundcolor=\color{gray!10},
    commentstyle=\color{green!60!black},
    keywordstyle=\color{blue},
    stringstyle=\color{red},
    captionpos=b,
    xleftmargin=2em,
    framexleftmargin=1.5em
}

\input{pustaka/tanda-hubung.tex}

% Menambahkan resource daftar pustaka
\addbibresource{pustaka/pustaka.bib}

% Isi keseluruhan dokumen
\begin{document}
  % Nomor halaman pembuka dimulai dari sini
  \pagenumbering{roman}

  % Atur ulang penomoran halaman
  \setcounter{page}{1}

  % Sampul Bahasa Indonesia
  \newcommand\covercontents{sampul/konten-id.tex}
  \input{sampul/sampul-luar.tex}

  % Lembar pengesahan
  \chapter*{LEMBAR PENGESAHAN}

% Menyembunyikan nomor halaman
\thispagestyle{empty}

\begin{center}
  % Ubah kalimat berikut dengan judul tugas akhir
  \textbf{KALKULASI ENERGI PADA ROKET LUAR ANGKASA BERBASIS \emph{ANTI-GRAVITASI}}
\end{center}

\begingroup
  % Pemilihan font ukuran small
  \small

  \begin{center}
    % Ubah kalimat berikut dengan pernyataan untuk lembar pengesahan
    \textbf{PROPOSAL TUGAS AKHIR} \\
    Diajukan untuk memenuhi salah satu syarat memperoleh gelar
    Sarjana Teknik pada 
    Program Studi S-1 Teknik Dirgantara \\
    Departemen Teknik Dirgantara \\
    Fakultas Teknik Dirgantara \\
    Institut Teknologi Sepuluh Nopember
  \end{center}

  \begin{center}
    % Ubah kalimat berikut dengan nama dan NRP mahasiswa
    Oleh: \textbf{Elon Reeve Musk} \\
    NRP. 0123 20 4000 0001
  \end{center}

  \begin{center}
    Disetujui oleh Tim Penguji Proposal Tugas Akhir:
  \end{center}

  \vspace{10ex}

  \begingroup
    % Menghilangkan padding
    \setlength{\tabcolsep}{0pt}

    \noindent
    \begin{tabularx}{\textwidth}{X c}
      % Ubah kalimat-kalimat berikut dengan nama dan NIP dosen pembimbing pertama
      Nikola Tesla, S.T., M.T.          & \\
      NIP: 18560710 194301 1 001        & (Pembimbing)\\
      &  \\                             
      &  \\ 
      &  \\                             
      % Ubah kalimat-kalimat berikut dengan nama dan NIP dosen pembimbing kedua
      Wernher von Braun, S.T., M.T.     & \\
      NIP: 19230323 197706 1 001        & (Ko-Pembimbing)\\
      &  \\                             & \\
      &  \\
      &  \\
      % Ubah kalimat-kalimat berikut dengan nama dan NIP dosen penguji pertama
      Dr. Galileo Galilei, S.T., M.Sc.  & \\
      NIP: 15640215 164201 1 001        & (Penguji)\\
      &  \\                             & \\
      &  \\
      &  \\
      % Ubah kalimat-kalimat berikut dengan nama dan NIP dosen penguji kedua
      Friedrich Nietzsche, S.T., M.Sc.  & \\
      NIP: 18441015 190008 1 001        & (Penguji II)\\
      &  \\                             & \\ 
      &  \\
      &  \\
      % Ubah kalimat-kalimat berikut dengan nama dan NIP dosen penguji ketiga
      Alan Turing, ST., MT.             & \\
      NIP: 19120623 195406 1 001        & (Penguji III)\\
    \end{tabularx}
  \endgroup

  \vspace{4ex}

  \begin{center}
    % Ubah text dibawah menjadi tempat dan tanggal
    \textbf{SURABAYA} \\
    \textbf{Mei, 2077}
  \end{center}
\endgroup

  \cleardoublepage

  % Abstrak
  \chapter*{ABSTRAK}
\begin{center}
  \large
  \textbf{KALKULASI ENERGI PADA ROKET LUAR ANGKASA BERBASIS \emph{ANTI-GRAVITASI}}
\end{center}
\addcontentsline{toc}{chapter}{ABSTRAK}
% Menyembunyikan nomor halaman
\thispagestyle{empty}

\begin{flushleft}
  \setlength{\tabcolsep}{0pt}
  \bfseries
  \begin{tabular}{ll@{\hspace{6pt}}l}
  Nama Mahasiswa / NRP &:& Elon Reeve Musk / 50252X1XXX\\
  Departemen &:& Teknik Informatika FTEIC - ITS\\
  Dosen Pembimbing &:& Nikola Tesla, S.T., M.T.\\
  Dosen Ko-Pembimbing &:& Wernher von Braun, S.T., M.T.\\
  \end{tabular}
  \vspace{4ex}
\end{flushleft}
\textbf{Abstrak}

% Isi Abstrak
Abstrak harus berisi seratus hingga dua ratus kata. \lipsum[1]

\vspace{2ex}
\noindent
\textbf{Kata Kunci: \emph{Roket, Anti-gravitasi, Meong}}
  \cleardoublepage

  \chapter*{ABSTRACT}
\begin{center}
  \large
  \textbf{\emph{ANTI-GRAVITY} BASED ENERGY CALCULATION ON OUTER SPACE ROCKETS}
\end{center}
% Menyembunyikan nomor halaman
\thispagestyle{empty}

\begin{flushleft}
  \setlength{\tabcolsep}{0pt}
  \bfseries
  \begin{tabular}{lc@{\hspace{6pt}}l}
  Student Name / NRP &: &Elon Reeve Musk / 50252X1XXX\\
  Department &: &Informatics FTEIC - ITS\\
  Advisor &: &Nikola Tesla, S.T., M.T.\\
  Co-Advisor &: &Wernher von Braun, S.T., M.T.\\
  \end{tabular}
  \vspace{4ex}
\end{flushleft}
\textbf{Abstract}

% Isi Abstrak
The abstract must consist between two hundred to three hundred words. \lipsum[1]

\vspace{2ex}
\noindent
\textbf{Keywords: \emph{Rocket, Anti-gravity, Meong}}
  \cleardoublepage

  \begin{spacing}{1}
    % Daftar Isi
    \phantomsection
    \renewcommand*\contentsname{DAFTAR ISI}
    \addcontentsline{toc}{chapter}{\contentsname}
    \tableofcontents
    \cleardoublepage

    % Daftar gambar
    \phantomsection
    \renewcommand*\listfigurename{DAFTAR GAMBAR}
    \addcontentsline{toc}{chapter}{\listfigurename}
    \listoffigures
    \cleardoublepage

    % Daftar tabel
    \phantomsection
    \renewcommand*\listtablename{DAFTAR TABEL}
    \addcontentsline{toc}{chapter}{\listtablename}
    \listoftables
    \cleardoublepage

    % Daftar kode sumber
    \phantomsection
    \renewcommand*\lstlistlistingname{DAFTAR KODE SUMBER}
    \addcontentsline{toc}{chapter}{\lstlistlistingname}
    \lstlistoflistings
    \cleardoublepage
  \end{spacing}

  % Nomor halaman isi dimulai dari sini
  \pagenumbering{arabic}

  % Konten pendahuluan
  \chapter{PENDAHULUAN}

\section{Latar Belakang}

% Ubah paragraf-paragraf berikut sesuai dengan latar belakang dari tugas akhir
Pesatnya perkembangan roket yang merupakan \lipsum[2]

\lipsum[3]

\section{Rumusan Masalah}

% Ubah paragraf berikut sesuai dengan rumusan masalah dari tugas akhir
Berdasarkan hal yang telah dipaparkan di latar belakang, \lipsum[4]

\section{Batasan Masalah}

\lipsum[6]

\section{Tujuan}

% Ubah paragraf berikut sesuai dengan tujuan penelitian dari tugas akhir
Tujuan dari penelitian ini adalah \lipsum[7][1-14]

\section{Manfaat}

% Ubah paragraf berikut sesuai dengan tujuan penelitian dari tugas akhir
Manfaat dari penelitian ini adalah sebagai berikut.

\subsection{Manfaat Teoritis}
\lipsum[8][1-14]

\subsection{Manfaat Praktis}
\lipsum[9][1-14]

  \cleardoublepage

  % Konten tinjauan pustaka
  \chapter{TINJAUAN PUSTAKA}

% Ubah konten-konten berikut sesuai dengan isi dari tinjauan pustaka
\section{Hasil Penelitian Terdahulu}
\lipsum[3]

\section{Dasar Teori}

\subsection{Hukum Newton}

% Contoh penggunaan referensi dari pustaka
Newton pernah merumuskan \parencite{Newton1687} bahwa \lipsum[8]
% Contoh penggunaan referensi dari persamaan
Kemudian menjadi persamaan seperti pada persamaan \ref{eq:FirstLaw}.

% Contoh pembuatan persamaan
\begin{equation}
  % Label referensi dari persamaan yang dibuat
  \label{eq:FirstLaw}
  % Baris kode persamaan yang dibuat
  \sum \mathbf{F} = 0\; \Leftrightarrow\; \frac{\mathrm{d} \mathbf{v} }{\mathrm{d}t} = 0.
\end{equation}

\lipsum[9]

\subsection{Anti Gravitasi}

\lipsum[10]

  \cleardoublepage

  % Konten metodologi
  \chapter{METODOLOGI}

% Ubah konten-konten berikut sesuai dengan isi dari metodologi

\section{Metode yang Digunakan}

\lipsum[11]

% Contoh input gambar dengan format *.jpg
\begin{figure} [H] \centering
  % Nama dari file gambar yang diinputkan
  \includegraphics[scale=0.45]{gambar/blueprint.jpg}
  % Keterangan gambar yang diinputkan
  \caption{\emph{Blueprint} roket yang akan diuji coba \parencite{SpaceXBlueprint}}
  % Label referensi dari gambar yang diinputkan
  \label{fig:Blueprint}
\end{figure}

% Contoh penggunaan referensi dari gambar yang diinputkan
Pada \emph{blueprint} yang tertera di Gambar \ref{fig:Blueprint}. \lipsum[12]

\section{Bahan dan Alat yang Digunakan}

Beberapa bahan dan peralatan yang digunakan untuk mendukung penelitian ini antara lain:

\subsection{Perangkat Keras}
\lipsum[13]

\subsection{Perangkat Lunak}
\lipsum[14]

\section{Urutan Pelaksanaan Penelitian}

\lipsum[17]

  \cleardoublepage

  % Konten jadwal penelitian
  \chapter{JADWAL PENELITIAN}

% Ubah tabel berikut sesuai dengan isi dari rencana kerja
\newcommand{\w}{}
\newcommand{\G}{\cellcolor{gray}}

Pada \emph{timeline} yang tertera di Tabel \ref{tbl:timeline} \lipsum[10]

\begin{table}[H]
  \captionof{table}{Tabel timeline}
  \label{tbl:timeline}
  \begin{tabular}{|p{3.5cm}|c|c|c|c|c|c|c|c|c|c|c|c|c|c|c|c|}

    \hline
    \multirow{2}{*}{Kegiatan} & \multicolumn{16}{|c|}{Minggu}                                                                       \\
    \cline{2-17}              &
    1                         & 2                             & 3  & 4  & 5  & 6  & 7  & 8  & 9  & 10 & 11 & 12 & 13 & 14 & 15 & 16 \\
    \hline

    % Gunakan \G untuk mengisi sel dan \w untuk mengosongkan sel
    Pengambilan data          &
    \G                        & \G                            & \G & \G & \w & \w & \w & \w & \w & \w & \w & \w & \w & \w & \w & \w \\
    \hline

    Pengolahan data           &
    \w                        & \w                            & \w & \w & \G & \G & \G & \G & \w & \w & \w & \w & \w & \w & \w & \w \\
    \hline

    Analisa data              &
    \w                        & \w                            & \w & \w & \w & \w & \w & \w & \G & \G & \G & \G & \w & \w & \w & \w \\
    \hline

    Evaluasi penelitian       &
    \w                        & \w                            & \w & \w & \w & \w & \w & \w & \w & \w & \w & \w & \G & \G & \G & \G \\
    \hline
  \end{tabular}
\end{table}

  \cleardoublepage

  % Daftar pustaka
  \chapter*{DAFTAR PUSTAKA}
  \addcontentsline{toc}{chapter}{DAFTAR PUSTAKA}
  \renewcommand\refname{}
  \vspace{2ex}
  \renewcommand{\bibname}{}
  \begingroup
    \def\chapter*#1{}
    \printbibliography
  \endgroup


\end{document}
